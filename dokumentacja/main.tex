\documentclass[eng,printmode]{mgr}
%opcje klasy dokumentu mgr.cls zostały opisane w dołączonej instrukcji

%poniżej deklaracje użycia pakietów, usunąć to co jest niepotrzebne
\usepackage{polski} %przydatne podczas składania dokumentów w j. polskim
%\usepackage[polish]{babel}%alternatywnie do pakietu polski, wybrać jeden z nich
\usepackage[utf8x]{inputenc} %kodowanie znaków, zależne od systemu
\usepackage[T1]{fontenc} %poprawne składanie polskich czcionek

%pakiety do grafiki
\usepackage{graphicx}
\usepackage{subfigure}
\usepackage{psfrag}
\usepackage{epstopdf}

\usepackage{indentfirst}
%pozycjonowanie grafiki
\usepackage{float}
\graphicspath{ {figures/} }

%pakiety dodające dużo dodatkowych poleceń matematycznych
\usepackage{amsmath}
\usepackage{amsfonts}

%pakiety wspomagające i poprawiające składanie tabel
\usepackage{supertabular}
\usepackage{array}
\usepackage{tabularx}
\usepackage{hhline}

%pakiet wypisujący na marginesie etykiety równań i rysunków zdefiniowanych przez \label{}, chcąc wygenerować finalną wersję dokumentu wystarczy usunąć poniższą linię
\usepackage{showlabels}

%pozostałe ustawienia
\usepackage[hidelinks,pagebackref=true,backref=false]{hyperref}
%\usepackage{lscape}
%\usepackage{pdflscape}
\usepackage{rotating}
\usepackage{changepage}

\usepackage{booktabs}

\topmargin -9mm

%ustawienia spisu treści
%\setcounter{tocdepth}{3}
%\setcounter{secnumdepth}{3}

%dane do złożenia strony tytułowej
\title{Badanie czasu pracy}
\engtitle{Work time guardian}
\author{Mikołaj Jermakowicz, Kamil Machnicki}
\supervisor{dr inż. Marek Woda}
%\date{2008} %standardowo u dołu strony tytułowej umieszczany jest bieżący rok, to polecenie pozwala wstawić dowolny rok

\field{Informatyka (INF)}
\specialisation{Inżynieria Internetowa (INT)}

\makeatletter
\hypersetup{
	pdfstartview={XYZ null null 1.00},
	pdfpagemode=UseNone,
	pdfauthor={\@author},
	pdftitle={\@title},
	pdfsubject={\@naglowek},
	pdfcreator={pdf\LaTeX}, 
	pdfproducer={TeXstudio}
}
\makeatother

\let\origappendix\appendix % save the existing appendix command
\renewcommand\appendix{\clearpage\pagenumbering{Roman}\origappendix}
\usepackage{etoolbox}
\apptocmd{\thebibliography}{\raggedright}{}{}

%\usepackage{spacje}

%definicje własnych poleceń
%\newcommand{\R}{I\!\!R} %symbol liczb rzeczywistych, działa tylko w trybie matematycznym
%\newtheorem{theorem}{Twierdzenie}[section] %nowe otoczenie do składania twierdzeń
\begin{document}
\bibliographystyle{plabbrv} %tylko gdy używamy BibTeXa, ustawia polski styl bibliografii

\maketitle %polecenie generujące stronę tytułową

\tableofcontents %spis treści

\chapter{Wstęp}

Jako zaliczenie kursu Zastosowania Systemów Wbudowanych należało zrealizować wybrany przez grupę projekt, dotyczący układów wbudowanych. Pracę wykonać można było wykorzystując platformy Raspberry Pi, Arduino, bądź Beagleboard. Przez naszą grupę wybrany został temat \emph{Badanie czasu pracy}, który zrealizowany został na płytce Raspberry Pi.

\section{Cel i zakres pracy}

Celem wybranego tematu było stworzenie projektu, który badałby godziny pracy pracowników danej placówki. Każdy pracownik przy wejściu oraz wyjściu odbijałby się przypisaną do siebie kartą, a system zliczałby godziny spędzone tego dnia i przedstawiał je w wygodnej formie dla administratora. Administrator mógłby dodawać pracowników oraz przypisać kartę do pracownika. W przypadku, gdy uprawniony użytkownik (taki, który znajduje się w bazie) odbije się w systemie, system poinformuje go o tym w sposób wizualny, a administrator będzie mógł zobaczyć w panelu dane pracownika i godzinę odbicia. W przypadku, gdy odbijana karta nie jest przypisana do żadnego pracownika, system w sposób wizualny zaalarmuje o tym.

\section{Funkcjonalności}

Lista funkcjonalności wymaganych do zrealizowania projektu:

\begin{itemize}
  \item Rozpoznawanie pracownika na podstawie odczytu z karty.
  \item Zapis godziny wejścia i wyjścia pracownika.
  \item Prezentacja danych zebranych w bazie danych.
  \item Sygnalizacja przepuszczenia pracownika i błędnej karty.
  \item Panel wyświetlania pracowników i godziny ich pracy.
  \item Panel dodawania nowych pracowników i przypisywania do nich karty.
\end{itemize}

\section{Moduły}

Aby zrealizować projekt, został on podzielony na dwa moduły:

\begin{enumerate}
  \item \textbf{Sterownik} - moduł obsługujący wczytywanie karty i zapisywanie danych do bazy, a także informowanie użytkownika o pomyślnej bądź niepomyślnej autoryzacji.
  \item \textbf{Aplikacja} - panel administratora, gdzie wyświetlane są godziny pracy pracowników oraz gdzie można zarządzać pracownikami i przypisywać do nich karty.
\end{enumerate}
\chapter{Sterownik}

Aby zrealizować zamierzony cel, wybrana została technologia zbliżeniowa RFID, gdyż jest ona obecnie bardzo popularna i można znaleźć wiele sprzętu w niej operujących.

\section{Wykorzystany sprzęt}

Jako system bazowy został wykorzystany Raspberry Pi w wersji 2 B v1.1 (Rysunek \ref{fig:raspberry}). Różni się on od poprzednich wersji płytki ilością pinów oraz wielkością pamięci operacyjnej i szybszym mikroprocesorem.

\begin{figure}[h!]
	\centering
	\includegraphics[width=\linewidth]{img/rasp.jpg}
	\label{fig:raspberry}
	\caption[Raspberry Pi w wersji 2 B v1.1]{Raspberry Pi w wersji 2 B v1.1\footnotemark}
\end{figure}
\footnotetext{https://en.wikipedia.org/wiki/Raspberry\_Pi}

Aby obsłużyć technologię RFID użyto modułu z anteną MF RC522 (Rysunek \ref{fig:rfid}). Pracuje on z częstotliwością 13,56 MHz, czyli najpopularniej występującą częstotliwością RFID. Moduł ten łączy się z Raspberry po interfejsie SPI.

\begin{figure}[h!]
	\centering
	\includegraphics[width=\linewidth]{img/RFID-RC522-pinout.png}
	\label{fig:rfid}
	\caption[Czytnik RFID z anteną MF RC522]{Czytnik RFID z anteną MF RC522\footnotemark}
\end{figure}
\footnotetext{https://github.com/r00tGER/RFID-RC522}

Aby złączyć moduły ze sobą wykorzystana została płytka stykowa, zwana inaczej płytką prototypową (Rysunek \ref{fig:stykowa}).

\begin{figure}[h!]
	\centering
	\includegraphics[width=\linewidth]{img/plytka-stykowa.jpg}
	\label{fig:stykowa}
	\caption[Płytka prototypowa na 830 styków]{Płytka prototypowa na 830 styków\footnotemark}
\end{figure}
\footnotetext{http://technovade.pl/arduino/arduino-akcesoria/plytka-stykowa-prototypowa-830-otworow.html}

Do sygnalizacji poprawnej bądź niepoprawnej autoryzacji, użyte zostały dwie zwykłe diody - czerwona do sygnalizacji błędu oraz zielona do sygnalizacji poprawnej autoryzacji. Do połączenia ich z płytką wykorzystano dwa rezystory, o oporności odpowiednio 220 Ohm oraz 150 Ohm.

\section{Podłączenie}

Przed przystąpieniem do realizacji należało najpierw przylutować czytnik RFID do goldpinów, gdyż w przeciwnym wypadku siła sygnału byłaby za słaba, aby móc przesyłać dane do oraz z powrotem z płytki. Po zlutowaniu podłączono wszystkie styki czytnika do pinów SPI, zgodnie z dokumentacją modułu MFRC522 (Pozycja \cite{mfrc}) oraz schematem rozłożenia wyjść na płytce (Pozycja \cite{piny}).

MFRC522 podłączony został do poniższych portów (pin IRQ nie jest wymagany do podłączenia, więc pozostał luźny):

\begin{table}[h!]
\centering
\caption{Połączenia pinów czytnika RFID}
\label{piny}
\begin{tabular}{@{}|c|c|c|@{}}
\toprule
\textbf{MFRC522} & \textbf{Numer pinu Raspberry} & \textbf{Nazwa pinu} \\ \midrule
SDA & 24 & GPIO8 \\ \midrule
SCK & 23 & GPIO11 \\ \midrule
MOSI & 19 & GPIO10 \\ \midrule
MISO & 21 & GPIO9 \\ \midrule
IRQ & x & x \\ \midrule
GND & 9 & GND \\ \midrule
RST & 22 & GPIO25 \\ \midrule
3.3V & 1 & 3V3 \\ \bottomrule
\end{tabular}
\end{table}

Diody podłączone zostały jak na Rysunku \ref{fig:led}, czyli katodę diody czerwonej podłączono do masy, zaś anodę poprzez 220-Ohmowy rezystor do pinu GPIO24. Dioda zielona podłączona została bardzo podobnie, z tą różnicą, że spięto ją z pinem GPIO23 oraz ze względu na większe natężenie prądu potrzebne do jej zaświecenia, zastosowano opornik 150-Ohmowy.

\begin{figure}[h!]
	\centering
	\includegraphics[width=0.5\linewidth]{img/led.png}
	\label{fig:led}
	\caption[Podłączenie czerwonej diody]{Podłączenie czerwonej diody}
\end{figure}

\newpage

Poniższy schemat z programu Fritzing\footnote{http://fritzing.org/home/} przedstawia finalny układ połączeń.

\begin{figure}[h!]
	\centering
	\includegraphics[width=\linewidth]{img/projekt_bb.png}
	\label{fig:projekt}
	\caption[Finalny wygląd połączeń]{Finalny wygląd połączeń}
\end{figure}

\section{Konfiguracja}

Aby zmusić kod do działania, należy skonfigurować potrzebne narzędzia i biblioteki. Należy najpierw uruchomić do działania porty SPI. Aby tego dokonać wpisujemy w terminalu:

\begin{verbatim}
$ sudo raspi-config
\end{verbatim}

Następnie przejść do Advanced Options i przestawić SPI na Enabled. Na starszych wersjach Raspberry należało również SPI z listy zablokowanych, poprzez uruchomienie edytora:

\begin{verbatim}
$ sudo nano /etc/modprobe.d/raspi-blacklist.conf
\end{verbatim}

I zakomentowanie linii dodając przed nią znak \#:

\begin{verbatim}
blacklist spi-bcm2708
\end{verbatim}

Kolejnym krokiem jest zrestartowanie urządzenia, i po restarcie wpisanie:

\begin{verbatim}
$ ls /dev/spidev0.*
\end{verbatim}

Powinniśmy otrzymać wynik podobny do następującego, co świadczy o tym, że SPI zostało uruchomione poprawnie:

\begin{verbatim}
/dev/spidev0.0 /dev/spidev0.1
\end{verbatim}

Kolejnym krokiem jest włączenie drzewa urządzeń. Należy wpisać:

\begin{verbatim}
$ sudo nano /boot/config.txt
\end{verbatim}

I dodać na końcu pliku następującą linijkę:

\begin{verbatim}
device_tree=on
\end{verbatim}

Następnie należy jeszcze zaktualizować sterownik dla mikrokontrolera Broadcom BCM 2835. Pobieramy jego najnowszą wersję \footnote{http://www.airspayce.com/mikem/bcm2835/} i wpisujemy kolejno:

\begin{verbatim}
$ tar zxvf bcm2835-1.xx.tar.gz
$ cd bcm2835-1.xx
$ ./configure
$ make
$ sudo make check
$ sudo make install
\end{verbatim}

Po tych krokach urządzenie jest już skonfigurowane do korzystania z SPI oraz modułu RFID.

\newpage

\section{Oprogramowanie}

Oprogramowanie napisane zostało w języku Python i wykorzystuje do działania dwie zewnętrzne biblioteki. Pierwszą jest SPI-Py\footnote{https://github.com/lthiery/SPI-Py}, która umożliwia korzystanie z interfejsu SPI z poziomu interpretera Python. Drugą jest MFRC522-python\footnote{https://github.com/mxgxw/MFRC522-python}, dzięki której można korzystać z modułu czytnika kart RFID.

W aplikacji wyróżnić można kilka modułów:

\begin{enumerate}
  \item \texttt{main.py} - główny moduł systemu, odpowiada za pętlę programu.
  \item \texttt{logger.py} - moduł odpowiedzialny za obsługę logów aplikacji.
  \item \texttt{database.py} - łączenie się z bazą danych oraz dodawanie do niej rekordów.
  \item \texttt{nfc.py} - obsługa odczytywania danych z karty.
\end{enumerate}

Program uruchamia się wpisując w konsoli (wymagana jest wersja Pythona co najmniej 3.0):

\begin{verbatim}
python3 main.py
\end{verbatim}

Aplikacja spróbuje teraz połączyć się z bazą danych i w wypadku pomyślnego połączenia uruchomiona zostanie główna pętla programu. Pętla ta uruchamia mniejszą pętlę w module \texttt{nfc.py}, który to w czasie ciągłym (aż do momentu wciśnięcia kombinacji klawiszy Ctrl + C) uruchamia cewkę na module MFRC522 i sprawdza czy w pobliżu nie znajduje się karta RFID. Jeśli tak, to odczytywany jest identyfikator karty, który następnie porównywany jest z identyfikatorami wszystkich przypisanych do systemu użytkownikami. Jeśli Id zostanie odnaleziony, do bazy zostaje dodany pod Id pracownika timestamp z aktualną datą i godziną odbicia się. System sygnalizuje też poprawną autoryzację miganiem zielonej diody. Jeśli identyfikator nie zostanie odnaleziony w bazie, system sygnalizuje to miganiem diody czerwonej.

Aplikacja kończy swe działanie w momencie wykrycia kombinacji klawiszy Ctrl + C, kończąc działanie modułu MFRC522 i zamykając połączenie do bazy danych.
\chapter{Aplikacja}
\section{Program prezentacji danych}
Aplikacja została wykonana z myślą o pracodawcy, który chciałby mieć podgląd obecnych pracowników w biurze, oraz podsumowanie przepracowanego czasu dla poszczególnego pracownika.
\subsection{Podgląd obecnych pracowników}
W tej części programu można podejżeć obecnych pracowników ich imię, nazwisko oraz czas przybycia.
Widok odświeża się do pół sekundy.
\begin{figure}[h!]
	\centering
	\includegraphics[]{img/present_employees.jpg}
	\label{fig:present_employees}
	\caption[Obecni pracownicy]{Obecni pracownicy}
\end{figure}
\subsection{Podgląd prepracowanego czasu pracowników}
Podgląd umożliwia wybrania przedziału czasu. Nastepnie wyświetli się imie, nazwisko oraz przepracowany czas.
\begin{figure}[h!]
	\centering
	\includegraphics[]{img/time_worked.jpg}
	\label{fig:time_worked}
	\caption[Przepracowany czas]{Przepracowany czas}
\end{figure}
\newpage
\subsection{Dodawanie pracowników}
Ta część programu pozwala na dodawanie użytkowników. Wymagane są wszystkie pola do wpsiania.
Należy podać: Imię, Nazwisko, id taga rfid. email oraz hasło do systemu.
\begin{figure}[h!]
	\centering
	\includegraphics[]{img/add_user.jpg}
	\label{fig:add_user}
	\caption[Dodaj pracownika]{Dodaj pracownika}
\end{figure}
\section{Technologie}
Wykorzystane technologie:
\begin{itemize}
\item Python3.4 - Język programowania, w którym została wykonana aplikacja.
\item PyQt5 - Biblioteka języka python do interfejsu graficznego.
\item psycopg2 - Biblioteka języka python do komunikacji z bazą danych PostgreSQL
\item PostgreSQL - Baza danych.
\item Vagrant - Środowisko do przygotowania maszyn wirtualnych.
\item Raspbian - Dystrybujca linuxa przygotowanego dla RPi na podstawie dystrybucji Debian.
\end{itemize}
\section{Przygotowanie środowiska}
Aby uruchomić program prezentacji danych potrzeba zainstalować interpreter Python3.4, oraz biblioteki PyQt5, psycopq2. Program używa danych przetrzymywanych w bazie danych PostgreSQL.
\section{Baza Danych}
Baza danych składa się z dwóch tabel, Pracowników oraz przejść. Do każdego pracownika może być przypisane wiele przejść.

\begin{figure}[h!]
	\centering
	\includegraphics[width=\linewidth]{img/ERD_screen.jpg}
	\label{fig:ERD}
	\caption[Diagram ERD]{Diagram ERD}
\end{figure}
Możliwe jest ustawinie bazy danych na dowolonym urządzeniu. Takie rozwiązanie było niezbedne ze względu na to, że posiadaliśmy tylko jeden komputer RPi. Aplikacja prezentacji danych, komunikuje się wyłącznie z bazą danych, więc możliwe było tworzenie aplikacji bez komputera RPi.
\subsection{Baza danych na maszynie wirtualnej}
W ramach projektu został przygotowany plik konfiguracyjny maszyny\\ (github:/database/Vagrantfile). Aby wygenerować maszynę wirtualną potrzebny jest program Vagrant. Uruchomienie skonfigurowanej maszyny wirtualenej, polega na wejściu do katalogu z plikiem konfigurayjnym i wywołaniem komędy "vagrant up". Następnie trzeba przekazać port 5432 bazy danych z maszyny wirtualnej do systemu operacyjnnego na ip 127.0.0.1, w projekcie wykorzystywany był do tego program putty.
Dane logowania do maszyny wirtualnej to: użytkownik - vagrant, hasło - vagrant.
\subsection{Baza dnych na Raspberry Pi}
Do pracy na RPi Wybraliśmy dystrybucje linuxa Raspbian.
Aby skonfigurować bazę danych na Raspberry Pi należy wykonać następujące komendy:
\lstset{language=bash, breaklines=true}
\begin{lstlisting}
apt-get update
apt-get install postgresql postgresql-contrib postgis gpsbabel git libsqlite3-dev libreadline-dev libpq-dev libbz2-dev zlib1g-dev libpqxx-dev libzip-dev -y
echo -ne "alamakota\nalamakota" | su - postgres -c 'createuser -P -e wbudowane'
su - postgres -c 'createdb -e -O wbudowane wbudowane'
su - postgres -c 'psql wbudowane < /vagrant/czaspracyBD.sql'
su - postgres -c 'psql wbudowane -c "GRANT ALL ON TABLE Employee TO wbudowane;"'
su - postgres -c 'psql wbudowane -c "GRANT ALL ON TABLE Passage TO wbudowane;"'
su - postgres -c 'psql wbudowane -c "GRANT USAGE, SELECT ON ALL SEQUENCES IN SCHEMA public to wbudowane;"'
\end{lstlisting}




\chapter{Podsumowanie}
W ramach projektu Zostały zaimplementowane wszystkie funkcjonalności podstawowe jakei zakładaliśmy:
\begin{itemize}
\item Rozpoznanie pracownika na podstawie odczytu z karty
\item Zapis godziny wejścia i wyjścia pracownika
\item Przentacja danych zebranych w bazie danych
\item Sygnalizacja przepuszczenia pracownika i błędnej karty
\end{itemize}
\section{Ocena krytyczna}
Nie udało się zrealizować funkcjonalności opcjonalnych:
\begin{itemize}
\item Hasło wpisywane z klawiatury numerycznej
\item Analiza zebranych danych
\item Obsługa niepoprawnego zachowania użytkownika
\end{itemize}
Aplikacja prezentacji przedstawia jedynie surowe dane, oraz nie została zaimplementowana obsługa błędów
\section{Możliwości rozwoju}
System można rozszerzyć między innymi o funkconalności opcjonalne. Przydatne mogło by być generowanie  raportów z analizą danych. Innym przydatnym rozszerzniem mogłoby być wpisywanie urlopów oraz uwzględnianie ich w raportach.

\nocite{*}
\newpage
\phantomsection
\addcontentsline{toc}{chapter}{\bibname}
\bibliography{bibliografia}

\newpage
\phantomsection
\addcontentsline{toc}{chapter}{\listfigurename}
\listoffigures

\newpage
\phantomsection
\addcontentsline{toc}{chapter}{\listtablename}
\listoftables

\end{document}

